\documentclass[letterpaper,10pt]{article}
\usepackage[T1]{fontenc}
\usepackage[utf8]{inputenc}
\usepackage{lmodern}
\usepackage[english]{babel}
\usepackage{amsmath}
\usepackage{amsfonts}
% \usepackage{fullpage}
\usepackage{enumerate}
\usepackage{setspace}
\usepackage{ amssymb }
\usepackage{mathtools}
\usepackage{fancyhdr}
\usepackage{algpseudocode}
\pagestyle{fancy}

\lhead{Assignment 2 - CSC265}
\rhead{David Knott, \#999817685}

\DeclarePairedDelimiter{\ceil}{\lceil}{\rceil}
\DeclarePairedDelimiter{\floor}{\lfloor}{\rfloor}

% http://www.maths.tcd.ie/~dwilkins/LaTeXPrimer/Theorems.html
\newtheorem{theorem}{Theorem}[section]
\newtheorem{lemma}[theorem]{Lemma}
\newtheorem{proposition}[theorem]{Proposition}
\newtheorem{corollary}[theorem]{Corollary}

\newenvironment{proof}[1][Proof]{\begin{trivlist}
\item[\hskip \labelsep {\bfseries #1}]}{\end{trivlist}}
\newenvironment{definition}[1][Definition]{\begin{trivlist}
\item[\hskip \labelsep {\bfseries #1}]}{\end{trivlist}}
\newenvironment{example}[1][Example]{\begin{trivlist}
\item[\hskip \labelsep {\bfseries #1}]}{\end{trivlist}}
\newenvironment{remark}[1][Remark]{\begin{trivlist}
\item[\hskip \labelsep {\bfseries #1}]}{\end{trivlist}}

\newcommand{\qed}{\nobreak \ifvmode \relax \else
      \ifdim\lastskip<1.5em \hskip-\lastskip
      \hskip1.5em plus0em minus0.5em \fi \nobreak
      \vrule height0.75em width0.5em depth0.25em\fi}


% \def\pball{\mathrel{%
%     \mathchoice{\PBALL}{\PBALL}{\scriptsize\PBALL}{\tiny\PBALL}%
% }}
% \def\PBALL{{%
%     \setbox0\hbox{B}%
%     \rlap{\hbox to \wd0{\hss|\hss}}\box0
% }}

\allowdisplaybreaks
\newcommand{\tinyquot}[1]{\begin{center}{\footnotesize #1}\end{center}}
\newcommand*{\QEDA}{\hfill\ensuremath{\blacksquare}}
\addtolength{\belowdisplayskip}{-25mm}
\begin{document}
\title{Assignment 2 - CSC265}
\author{David Knott \\  Student \#999817685}
\date{October 21, 2013}
\maketitle
\begin{enumerate}
	\item 

	\clearpage
	\item 
	\begin{enumerate}[a)]
		\item Note that for any $h \in \mathcal{H}$ for all distinct $x_1,x_2,x_3 \in U$ and all $y_1,y_2,y_3 \in \{0,\dots,m-1\}$ we have that:
		\begin{align*}
		\text{Prob}_{h\in\mathcal{H}}[ h(x_1) = y_1 \wedge h(x_2) = y_2 \wedge h(x_3) = y_3 ] & = \\
		\text{Prob}_{h\in\mathcal{H}}[h(x_1) = y_1] & \times \\
		\text{Prob}_{h\in\mathcal{H}}[h(x_2) = y_2] & \times \\
		\text{Prob}_{h\in\mathcal{H}}[h(x_3) = y_3] & = \frac{1}{m^3}
		\end{align*}
		Since the $x$s are arbitrary, all of these probabilities must be the same. Solving for each probability yields $1/m$. So for for all distinct $x_1,x_2,x_3 \in U$ and all $y_1,y_2,y_3 \in \{0,\dots,m-1\}$ we have that 
		\begin{align*}
		\text{Prob}_{h\in\mathcal{H}}[h(x_1) = y_1] & = \frac{1}{m} \\
		\text{Prob}_{h\in\mathcal{H}}[h(x_2) = y_2] & = \frac{1}{m} \\
		\text{Prob}_{h\in\mathcal{H}}[h(x_3) = y_3] & = \frac{1}{m}
		\end{align*}
		If we let $y_1 = h(x_2)$, since $h(x_2) \in \{0,\dots,m-1 \}$ we have:
		$$ \text{Prob}_{h\in\mathcal{H}}[h(x_1) = h(x_2)] = \frac{1}{m} $$
		Which implies that a 3-universal family is universal.
		\item 
	\end{enumerate}
	\clearpage

	\item
	\clearpage

	\item
	\clearpage

	\item
	\clearpage
	\item 
		For a list of lists with these restrictions, the number of lists and the maximum size of the largest list will be:
		$$ \ceil{ \frac{1}{2} (\sqrt{8n + 1} - 1)} \in O(\sqrt{n}) $$
		In addition to this data structure, each group will keep track of how ``rotated'' their list is. This way we can find the maximum and minimum elements of an array in constant time. Keeping this information updated will not add anything to the runtime of any of the algorithms. For convenience, if $g$ is a subgroup of $S$, then $g.\max$ and $g.\min$ will yield the maximum and minimum elements.

	\begin{enumerate}[a)]
		\item For insertion, for each step in the algorithm we maintain the property that the $k^{th}$ group has at most $k$ elements. This means every time we insert an element into a full group we must take one out. Taking advantage of this fact, consider the following sub algorithm:

		\begin{algorithmic}
		\Function{INSERT-INTO-GROUP}{$g$, $x$}
			\If{$x \geq g.\max$}
				\State \Return $x$
			\EndIf

			\State $r \gets g.\max$
			\State Put $x$ where $g.\max$ was.
			\State Subtract one to the group's rotation amount.
			\State Bubble $x$ up the list until it reaches equilibrium.
			\State \Return $r$

		\EndFunction
		\end{algorithmic}
		Since all the elements of a group are smaller than any of the elements in it's successor, the return value of this algorithm will be the minimum of the next group if we inserted it there. This is precisely how the algorithm will work: 
		\begin{algorithmic}
		\Function{INSERT}{$S$, $x$}
			\State $i \gets 1$
			\State $r \gets x$
			\While{$i < |S|$}
				\State $r \gets \text{INSERT-INTO-GROUP}(S[i], r)$
			\EndWhile

			\If{$S[|S|]$ is full}
				\State $S[|S|+1] \gets [r]$
			\Else
				\State Put $r$ at the end of $S[|S|]$ and bubble until equilibrium.
			\EndIf
		\EndFunction
		\end{algorithmic}
		The subfunction INSERT-INTO-GROUP has two behaviors: the first will return $x$ without affecting $g$, the second will return the maximum value of $g$, having inserted $x$ into $g$ by preforming at most $|g|$ shifting operations. If $x$ was already the minimum of $g$ then there would be only a constant amount of shifting.

		Through the execution of the algorithm, once calls to INSERT-INTO-GROUP start exhibiting the second behavior, they will continue to do so for the rest of the algorithm. This is because the max of one group is the min of the next. Additionally, INSERT-INTO-GROUP will take $O(|g|)$ time once, and $O(1)$ time anytime else. Since INSERT INTO GROUP is called $\sqrt{n}$ times, this means INSERT will take $O(\sqrt{n})$ time.

		\item For deletions we assume that the input $x$ is the position in the list of all lists where the element we're removing is. The algorithm is constructed as such:

		\begin{algorithmic}
		\Function{DELETE}{$S$, $x$}
			\If{$x \in S[|S|]$}
				\State Remove $x$ from $S[|S|]$ and shift everything backward one space.
				\State Subtract one from the rotation amount if $x$ was a minimum.
				\State \Return
			\Else
				\State $r \gets S[|S|].\min$
				\State Remove the minimum from $S[|S|]$, shift everything backward.
				\State $i \gets |S|-1$
				\While{$i > 0$}
					\If{$x \in S[i]$}
						\State Replace $x$ with $r$.
						\State Bubble $r$ it until it reaches it's place.
						\State Change the rotation amount of $S[i]$ accordingly.
						\State \Return
					\Else
						\State $m \gets r$
						\State $r \gets S[i].\min$
						\State Replace $S[i].\min$ with $m$.
						\State Subtract one from $S[i]$'s rotation amount.
					\EndIf
				\EndWhile
			\EndIf
		\EndFunction
		\end{algorithmic}
		What the algorithm essentially does is remove the element at position $x$ and replace it with the next group's minimum, doing necessary shifting to keep the current group sorted. It then recurses to the next group and replaces it's missing space with the next group's minimum. This takes constant time. It continues to do this until it reaches the bottom. The pseudo code actually does all of this in reverse, but explaining is better this way.

		This algorithm does at most $\sqrt{n}$ iterations of it's while loop. Only two iterations of this loop will take more than constant time. These iterations take $O(\sqrt{n})$ time. This means the algorithm takes $O(3 \sqrt{n}) = O(\sqrt{n})$ time.

		\item Searching can be done by doing a binary search over each group's max and min. If the search key ``falls between the cracks'' of two groups (i.e.\ for two successive groups $g_1$ and $g_2$ we find $g_1.\max < x < g_2.\min$) then it does not exist in the data structure. If $x \in [g.\min, g.\max]$ for some $g \in S$ then we do a binary search over the elements of $g$, keeping in mind that it's rotated by a set amount.
		\begin{algorithmic}
		\Function{SEARCH}{$S$, $x$}
			\State $intervalmin \gets 1$
			\State $intervalmax \gets 2|S|$
			\While{$intervalmin < intervalmax$}
				\State $midpoint$
			\EndWhile
		\EndFunction
		\end{algorithmic}

		Since there are at most $\sqrt{n}$ groups and at most $\sqrt{n}$ elements in each group, this algorithm takes $O(\log(\sqrt{n}))+O(\log(\sqrt{n})) = O(1/2 \log(n)) = O(\log(n))$ time.

		\item Implementing this structure as a dictionary is as simple as making each element of each array a pointer to a key-value pair. In comparison to other data structures, consider the following table:

		\begin{tabular}{|l|c|c|c|c|}
		  \hline
		   & Unsorted Ary & Sorted Ary & Red-Black & Rotated Lists \\
		  \hline \hline
		  Insert & $O(1)$ & $O(n)$ & $O(\log(n))$ & $O(\sqrt{n})$ \\
		  \hline
		  Delete & $O(1)$ & $O(n)$ & $O(\log(n))$ & $O(\sqrt{n})$ \\
		  \hline
		  Search & $O(n)$ & $O(\log(n))$ & $O(\log(n))$ & $O(\log(n))$ \\
		  \hline
		\end{tabular}

		Running time for search trumps all of these algorithms. However, insert and delete, although faster than the sorted array, is slower than Red-Black trees. This structure would be a good fit for situations where insertions and deletions are less common than searches.

		It would also be good for large datasets, because in practice the algorithm searches over a sorted list of size $\sqrt{n}$ and then a sorted list of size $n \leq \sqrt{n}$. If keys are ranked in such a way that lower values are searched more often, then $n$ would be small, and the amount of comparisons would be closer to $\log(\sqrt{n}) + \log(c)$ for some constant $c$.

		Red-Black trees don't have this advantage, since low valued keys would be closer to the leftmost side of the tree, and possibly near the bottom. This would make the practical number of comparisons around $\log(n)$ for low valued keys.

	\end{enumerate}

\end{enumerate}
\end{document}