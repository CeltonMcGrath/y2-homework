\documentclass[letterpaper,10pt]{article}
\usepackage[T1]{fontenc}
\usepackage[utf8]{inputenc}
\usepackage{lmodern}
\usepackage[english]{babel}
\usepackage{amsmath}
\usepackage{amsfonts}
\usepackage{fullpage}
\usepackage{graphicx}
\usepackage{enumerate}
\usepackage{setspace}
\usepackage{ amssymb }

% http://www.maths.tcd.ie/~dwilkins/LaTeXPrimer/Theorems.html
\newtheorem{theorem}{Theorem}[section]
\newtheorem{lemma}[theorem]{Lemma}
\newtheorem{proposition}[theorem]{Proposition}
\newtheorem{corollary}[theorem]{Corollary}

\newenvironment{proof}[1][Proof]{\begin{trivlist}
\item[\hskip \labelsep {\bfseries #1}]}{\end{trivlist}}
\newenvironment{definition}[1][Definition]{\begin{trivlist}
\item[\hskip \labelsep {\bfseries #1}]}{\end{trivlist}}
\newenvironment{example}[1][Example]{\begin{trivlist}
\item[\hskip \labelsep {\bfseries #1}]}{\end{trivlist}}
\newenvironment{remark}[1][Remark]{\begin{trivlist}
\item[\hskip \labelsep {\bfseries #1}]}{\end{trivlist}}

\newcommand{\qed}{\nobreak \ifvmode \relax \else
      \ifdim\lastskip<1.5em \hskip-\lastskip
      \hskip1.5em plus0em minus0.5em \fi \nobreak
      \vrule height0.75em width0.5em depth0.25em\fi}


% \def\pball{\mathrel{%
%     \mathchoice{\PBALL}{\PBALL}{\scriptsize\PBALL}{\tiny\PBALL}%
% }}
% \def\PBALL{{%
%     \setbox0\hbox{B}%
%     \rlap{\hbox to \wd0{\hss|\hss}}\box0
% }}

\allowdisplaybreaks
\newcommand{\tinyquot}[1]{\begin{center}{\footnotesize #1}\end{center}}
\newcommand*{\QEDA}{\hfill\ensuremath{\blacksquare}}
\addtolength{\belowdisplayskip}{-25mm}
\begin{document}
\title{Assignment 5 - MAT257}
\author{David Knott \\  Student \#999817685}
\date{October 18, 2013}
\maketitle
\begin{enumerate}

	\item Munkres \S8, \#2
	\begin{enumerate}
		\item Suppose that $f(x,y) = f(a,b)$ where $y,b \in (0, 2\pi)$. Then we have:
		\begin{align*}
		f(x,y) &= f(a,b) \implies \\
		\|f(x,y)\| &= \|f(a,b)\| \implies \\
		\sqrt{(e^x \cos y)^2 + (e^x \sin y)^2} & = \sqrt{(e^a \cos b)^2 + (e^a \sin b)^2} \implies \\
		e^{2x} (\cos^2 y + \sin^2 y) & = e^{2a} (\cos^2 b + \sin^2 b) \implies \\
		e^{2x} & = e^{2a} \implies \\
		x & = a
		\end{align*}
		Given $x = a$, the below follows:
		\begin{align*}
		f(x,y) &= f(a,b) \implies \\
		e^x \cos y &= e^a \cos b \implies \\
		\cos y &= \cos b 
		\end{align*}
		And:
		\begin{align*}
		f(x,y) &= f(a,b) \implies \\
		e^x \sin y &= e^a \sin b \implies \\
		\sin y &= \sin b
		\end{align*}
		Since $y$ and $b$ are in $(0, 2\pi)$ then this implies $y = b$. Since $f(x,y) = f(a,b) \implies (x, y) = (a, b)$ the function is one to one.

		\item $B = \mathbb{R}^2 - L$ where $L$ is the set $\{\; t(1,0) \;\big|\; t \in \mathbb{R}^{+} \;\}$.
		This is because for any $(x,y) \in \mathbb{R}^2$ if we can find a $\theta \in (0, 2\pi)$ an $r \in \mathbb{R}^{+}$ such that $(x, y) = (r \cos \theta, r \sin \theta)$ unless $x \in L$, since in that case $\theta = 0$ or $2\pi$. In the context of our function, for any $(r, \theta)$ we can get $(x, y)$ through the function $f(\log r, \theta)$. 

		\item By the inverse function theorem, if $Df(x,y)$ is nonsingular, then the function is one to one in some neighborhood of $(x,y)$ and the inverse function $g$ has the derivative $Dg(f(x,y)) = [Df(x,y)]^{-1}$. Taking $Df(x,y)$ for any $(x, y) \in \mathbb{R}^2$ yields:
		\begin{align*}
		Df(x,y) & =  \left[ \begin{matrix}
			\frac{\partial  f_1}{\partial x} & \frac{\partial  f_1}{\partial y}  \\
			\frac{\partial  f_2}{\partial x} & \frac{\partial  f_2}{\partial y} 
	 \end{matrix} \right] \\
		& =  \left[ \begin{matrix}
		e^x \cos y & -e^x \sin y \\
		e^x \sin y & e^x \cos y
	 \end{matrix} \right]
	 \end{align*}
	 The determinant of this matrix will be $(e^x \cos y)^2 + (e^x \sin y)^2 = e^{2x}$. Since $e^{2x} > 0$ for all $x$ this means the function's derivative is invertible anywhere.

	 Speaking generally again, if we take the formula for the inverse of a $2 \times 2$ matrix we find that $Df(x,y)^{-1}$ is:
		\begin{align*}
		\left[ \begin{matrix}
		e^x \cos y & -e^x \sin y \\
		e^x \sin y & e^x \cos y
	 \end{matrix} \right]^{-1} &=
		\frac{1}{e^{2x}} \left[ \begin{matrix}
		e^x \cos y & e^x \sin y \\
		-e^x \sin y & e^x \cos y
	 \end{matrix} \right] 
	 \end{align*}
	 To find $Dg(0,1)$ we must find an $(x, y)$ such that $f(x,y) = (0,1)$. Heuristically, we know that there exists a $\theta$ such that $\sin \theta = 1$ and $\cos \theta = 0$. If we let $x = 0$ and $y = \frac{\pi}{2}$ then $f(x,y) = (0, 1)$. Plugging the vector $(0, \pi/2)$ into our inverted matrix formula yields:
	 $$\frac{1}{e^{2x}}\left[ \begin{matrix}
		e^x \cos y & e^x \sin y \\
		-e^x \sin y & e^x \cos y
	 \end{matrix} \right] = 
	 \left[	\begin{matrix} 
	  0 & 1 \\
		-1 & 0
	 \end{matrix} \right] 
	  $$
	  Which is $Dg(0,1)$.
	\end{enumerate}

	\item Munkres \S8, \#3

	Expanding the function $f$ out into it's component functions:
	\begin{align*}
		f_1(x,y) & = x(x^2 + y^2) \\
		f_2(x,y) & = y(x^2 + y^2) 
	\end{align*}
	These are simple polynomials. This means $f$ is of class $C^{\infty}$

	Let $x, y \in B(0;1)$. Suppose $f(x) = f(y)$. Then $\|f(x)\| = \|f(y)\|$ and therefore:
	\begin{align*}
		\| \|x\|^2 \cdot x \| &=  \| \|y\|^2 \cdot y \| \implies \\
		\|x\|^2 \| x \| &=  \|y\|^2 \| y \| \implies \\
		\|x\| &=  \|y\| 
	\end{align*}
	If $\|x\| = \|y\| = 0$ then $x = y = 0$. Otherwise, we have $\|x\|^2 \cdot x = \|y\|^2 \cdot y$. Multiplying both sides by $1/\|x\|^2$ yields $x = y$. Therefore $f$ is one to one on the unit ball.

	For any $(x, y) \in \mathbb{R}^2$, $Df(x,y)$ will be the matrix:
	 $$\left[ \begin{matrix} 
	  3 x^2 + y^2 & 2xy \\
		2xy & 3y^2 + x^2
	 \end{matrix} \right] 
	  $$
	Since $f(0, 0) = 0$, the inverse function $g$ must have a derivative that is the inverse of $f$ at 0.  However, when $(x,y) = 0$ all of the terms in the matrix go to zero, and therefore it is not invertible. Hence, $g$ is not differentiable at 0.

	\item Munkres \S8, \#5
	

	\item Munkres \S9, \#3
	\item Munkres \S9, \#4
	\item Munkres \S9, \#5

\end{enumerate}
\end{document}
