\documentclass[letterpaper,10pt]{article}
\usepackage[T1]{fontenc}
\usepackage[utf8]{inputenc}
\usepackage{lmodern}
\usepackage[english]{babel}
\usepackage{amsmath}
\usepackage{amsfonts}
\usepackage{fullpage}
\usepackage{enumerate}
\usepackage{setspace}
\usepackage{ amssymb }

% http://www.maths.tcd.ie/~dwilkins/LaTeXPrimer/Theorems.html
\newtheorem{theorem}{Theorem}[section]
\newtheorem{lemma}[theorem]{Lemma}
\newtheorem{proposition}[theorem]{Proposition}
\newtheorem{corollary}[theorem]{Corollary}

\newenvironment{proof}[1][Proof]{\begin{trivlist}
\item[\hskip \labelsep {\bfseries #1}]}{\end{trivlist}}
\newenvironment{definition}[1][Definition]{\begin{trivlist}
\item[\hskip \labelsep {\bfseries #1}]}{\end{trivlist}}
\newenvironment{example}[1][Example]{\begin{trivlist}
\item[\hskip \labelsep {\bfseries #1}]}{\end{trivlist}}
\newenvironment{remark}[1][Remark]{\begin{trivlist}
\item[\hskip \labelsep {\bfseries #1}]}{\end{trivlist}}

\newcommand{\qed}{\nobreak \ifvmode \relax \else
      \ifdim\lastskip<1.5em \hskip-\lastskip
      \hskip1.5em plus0em minus0.5em \fi \nobreak
      \vrule height0.75em width0.5em depth0.25em\fi}


% \def\pball{\mathrel{%
%     \mathchoice{\PBALL}{\PBALL}{\scriptsize\PBALL}{\tiny\PBALL}%
% }}
% \def\PBALL{{%
%     \setbox0\hbox{B}%
%     \rlap{\hbox to \wd0{\hss|\hss}}\box0
% }}

\allowdisplaybreaks
\newcommand{\tinyquot}[1]{\begin{center}{\footnotesize #1}\end{center}}
\newcommand*{\QEDA}{\hfill\ensuremath{\blacksquare}}
\addtolength{\belowdisplayskip}{-25mm}
\begin{document}
\title{Assignment 3 - MAT257}
\author{David Knott \\  Student \#999817685}
\date{September 30, 2013}
\maketitle
\begin{enumerate}
	\item 
	\begin{enumerate}[a)]
		\item Since $A$ is closed this means $A^c$ will be open. This implies for every $x \notin A$ there is an epsilon-neighborhood around $x$ that does not intersect with $A$. This epsilon will be less than or equal to the distance between $x$ and any $y \in A$, since if the distance were less, then $y$ would be in the epsilon-neighborhood of $x$ that doesn't intersect with $A$ and that's a contradiction.
		\item Suppose there doesn't exist an $\epsilon > 0$. Consider the set $S_\epsilon = \{ x \in \mathbb{R}^n \ \big|\ \forall y \in B. d(x, y) < \epsilon \}$. If there's no epsilon, then $S_\epsilon \cap A \neq \emptyset$ for any $\epsilon > 0$. If we take the sequence $\{ x_n \}$ where $x_n$ is a point in the set $S_{1/n} \cap A$ then we have a convergent sequence to a point in the border of $B$. Since $A$ is closed, then the point that $\{x_n\}$ converges to must also be in $A$. But it's given that $B \cap A = \emptyset$. So we have a contradiction.
		% \item
	\end{enumerate}
	\item Since $K$ is compact, it must be closed and bounded. Since $K \subset U$, then all points $x$ in $K$ will have an epsilon-neighborhood around $x$ that's completely contained within $U$. Suppose $\epsilon_x$ denotes the size of the neighborhood around the point $x$ that satisfies the above condition. If we take the minimum of $\epsilon_x$ for all $x \in K$, call it $\epsilon_{min}$, then for all $x \in K$, $B(x, \epsilon_{min}) \subset U$.

	Notice that $\bigcup_{x \in K} B(x, \epsilon_{min})$ is an open set contained completely within $U$ that covers $K$. If we take the closure of that set, however, it might be that it's border lies outside of $U$. So, we prove the lemma that $\overline{B(x, \epsilon/2)} \subset B(x, \epsilon)$ for any $x$ or $\epsilon$.
	
	\begin{lemma}
		\label{ClosedBallInOpen}
		At any point $x$, the closed ball of size $\epsilon$ is contained within an open ball of size $\epsilon'$ if $\epsilon < \epsilon'$.
	\end{lemma}

	\begin{proof}
		Assume that $\epsilon < \epsilon'$. Since the norm of every point in the closed ball is less than or equal to $\epsilon$ and $\epsilon < \epsilon'$, then every point in the closed ball must be in the open ball of larger size.
	\end{proof}

	Using this lemma, we show that $\bigcup_{x \in K} B(x, \epsilon_{min})$ covers $\bigcup_{x \in K} \overline{B(x, \epsilon_{min}/2)}$ (I'll call these sets $A$ and $B$ respectively.) Since any $x \in B$ is in a closed ball of radius $\epsilon/2$ centered around some point in $K$, it must be in the open ball of radius $\epsilon$ around the same point, and thus in $A$. Therefore $B \subset A$. Furthermore, since $A \subset U$, then $B \subset U$. Also $B$ bounded, since each point is only $\epsilon/2$ distance away from the furthest point in the bounded set $K$. Therefore, $B$ is a closed and bounded set, which makes it compact. Also, the interior of $B$ covers $K$. Therefore for any closed set $K$ in an open set $U$, there exists a closed set in $U$ who's interior covers $K$.
	\item
	\begin{enumerate}[a)]
		\item To prove that this function is a metric on $\mathbb{R}^n$ we must show that it satisfies the metric properties. Namely, we show that:
		\begin{enumerate}[i.]
			\item $ \rho(x, y) = \rho(y, x) $
			\item $ \rho(x, y) \geq 0 $ with equality when $x = y$
			\item $ \rho(x, z) \leq \rho(x, y) + \rho(y, z) $
		\end{enumerate}
		To prove i., note that since $d(x,y) = d(y,x)$ we can simply do the following:
		\begin{align*}
			\rho(x, y) & = \frac{d(x,y)}{1+d(x,y)} \\
			& = \frac{d(y,x)}{1+d(y,x)} \\
			& = \rho(y,x)
		\end{align*}
		For ii., we once again base our proof on the fact that $\rho$ is simply a function of $d$. If $x = y$ then $d(x,y) = 0$, and hence $\rho(x,y) = \frac{0}{1} = 0$. If $x \neq y$ then $d(x,y) > 0$ and so $\rho(x,y)$ must be greater than zero since it's a positive number divided by a positive number.

		To prove iii., note that the function $f(x) = \frac{x}{x+1}$ is monotonically increasing for all $x \geq 0$. Also note that $f(x+y) \leq f(x) + f(y)$ since if $x = y = 0$ then $f(x+y) = f(x) + f(y) = 0$. If $x \neq 0$ and $y = 0$ then $f(x+y) = f(x) = f(x) + f(y)$. If $x, y \neq 0$ then the equation $\frac{xy(x+y+2)}{(x+1)(y+1)(x+y+1)}$ is positive, so the following holds:
		\begin{align*}
			f(x+y) & = \frac{x+y}{1+x+y} \\
			& \leq \frac{x+y}{1+x+y} + \frac{xy(x+y+2)}{(x+1)(y+1)(x+y+1)} \\
			& = \frac{x}{1+x} + \frac{y}{1+y} = f(x) + f(y)
		\end{align*}
		Also $f(d(x,y)) = \rho(x,y)$.

		Since $d(x,y) \geq 0$ for any $x, y \in \mathbb{R}^n$, we can apply $f$ to both sides of the inequality $ d(x, z) \leq d(x, y) + d(y, z) $ like so:
		\begin{align*}
			d(x, z) & \leq d(x, y) + d(y, z) \implies \\
			f(d(x, z)) & \leq f(d(x, y) + d(y, z)) \implies \\
			f(d(x, z)) & \leq f(d(x, y)) + f(d(y, z)) \implies \\
			\rho(x, z) & \leq \rho(x, y) + \rho(y, z)
		\end{align*}
		So $\rho$ is a metric on $\mathbb{R}^n$
		\item since $\rho(x,y) = f(d(x,y))$ and $d(x,y) \geq 0$ for all $x, y \in \mathbb{R}^2$ we simply show that $f$ is bounded for all $x \geq 0$. This is a simple task, since $f$ is a division of two positive numbers, where the denominator is larger than the numerator, $f$ must be always less than 1. Therefore, the metric $(\mathbb{R}^2, \rho)$ is bounded by 1.
		% \item For all Cauchy series $\{x_n\}$ in the metric space $(\mathbb{R}^n, \rho)$, if $\epsilon > 0$ then there exists an $N$ where for all $n,m > N$ we have $\rho(x_n,x_m) < \epsilon$.
	\end{enumerate}


\end{enumerate}
\end{document}