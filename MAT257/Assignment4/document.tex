\documentclass[letterpaper,10pt]{article}
\usepackage[T1]{fontenc}
\usepackage[utf8]{inputenc}
\usepackage{lmodern}
\usepackage[english]{babel}
\usepackage{amsmath}
\usepackage{amsfonts}
\usepackage{fullpage}
\usepackage{enumerate}
\usepackage{setspace}
\usepackage{ amssymb }

% http://www.maths.tcd.ie/~dwilkins/LaTeXPrimer/Theorems.html
\newtheorem{theorem}{Theorem}[section]
\newtheorem{lemma}[theorem]{Lemma}
\newtheorem{proposition}[theorem]{Proposition}
\newtheorem{corollary}[theorem]{Corollary}

\newenvironment{proof}[1][Proof]{\begin{trivlist}
\item[\hskip \labelsep {\bfseries #1}]}{\end{trivlist}}
\newenvironment{definition}[1][Definition]{\begin{trivlist}
\item[\hskip \labelsep {\bfseries #1}]}{\end{trivlist}}
\newenvironment{example}[1][Example]{\begin{trivlist}
\item[\hskip \labelsep {\bfseries #1}]}{\end{trivlist}}
\newenvironment{remark}[1][Remark]{\begin{trivlist}
\item[\hskip \labelsep {\bfseries #1}]}{\end{trivlist}}

\newcommand{\qed}{\nobreak \ifvmode \relax \else
      \ifdim\lastskip<1.5em \hskip-\lastskip
      \hskip1.5em plus0em minus0.5em \fi \nobreak
      \vrule height0.75em width0.5em depth0.25em\fi}


% \def\pball{\mathrel{%
%     \mathchoice{\PBALL}{\PBALL}{\scriptsize\PBALL}{\tiny\PBALL}%
% }}
% \def\PBALL{{%
%     \setbox0\hbox{B}%
%     \rlap{\hbox to \wd0{\hss|\hss}}\box0
% }}

\allowdisplaybreaks
\newcommand{\tinyquot}[1]{\begin{center}{\footnotesize #1}\end{center}}
\newcommand*{\QEDA}{\hfill\ensuremath{\blacksquare}}
\addtolength{\belowdisplayskip}{-25mm}
\begin{document}
\title{Assignment 4 - MAT257}
\author{David Knott \\  Student \#999817685}
\date{October 7, 2013}
\maketitle
\begin{enumerate}
	\item Munkres Question 1

	Since $f'(a; u) = Df(a) \cdot u$, then $f'(a; cu) = Df(a) \cdot cu = c(Df(a) \cdot u) = cf'(a; u)$.

	\item Munkres Question 2
	\begin{enumerate}[a)]
		\item Note that the function $f:\mathbb{R}^2 \to \mathbb{R}$ can be rewritten like so:
		\[
		 f((x,y)) =
		  \begin{cases}
		   \frac{xy}{\|(x,y)\|^2} & \text{if } (x, y) \neq 0 \\
		   0       & \text{if } (x, y) = 0
		  \end{cases}
		\]
		Let $u = (h, k)$ be an arbitrary, non-zero vector. The directional derivative of $f$ at origin with respect to $u$ will then be:
		\begin{align*}
			f'(0; u) & = \lim_{t \to 0} \frac{f(0 + tu) - f(0)}{t} \\
			& =  \lim_{t \to 0} \frac{thtk}{\|t(h,k)\|^2}\frac{1}{t} \\
			& =  \lim_{t \to 0} \frac{t^2hk}{t^3\|(h,k)\|^2} \\
			& =  \lim_{t \to 0} \frac{hk}{t\|(h,k)\|^2}
		\end{align*}
		This limit will only exist when $h$ or $k$ are zero. When this is the case, the limit will evaluate to zero.
		\item Since $D_1 f$ and $D_2 f$ are the directional derivatives of elementary unit vectors then $k$ or $h$ will be zero, and thus both of these will be zero.
		\item The function is not differentiable since some of the directional derivatives don't exist.
		\item The function is not continuous, since if $x = y$ then the limit of the function $\phi(x) = f((x,x)) = \frac{1}{2}$ as $x \to 0$ is $\frac{1}{2}$, however if $-x = y$ then the limit of the function $\phi(x) = f((x,-x)) = -\frac{1}{2}$ as $x \to 0$ is $-\frac{1}{2}$.
	\end{enumerate}

	\item Munkres Question 3
	\begin{enumerate}[a)]
		\item Note the function can be written as follows
		\[
		 f((x,y)) =
		  \begin{cases}
		   \frac{x^2y^2}{x^2y^2 + (y-x)^2)} & \text{if } (x, y) \neq 0 \\
		   0       & \text{if } (x, y) = 0
		  \end{cases}
		\]
		Let $u = (h, k)$ be an arbitrary, non-zero vector. The directional derivative of $f$ at origin with respect to $u$ will then be:
		\begin{align*}
			f'(0; u) & =  \lim_{t \to 0} \frac{f(0 + tu) - f(0)}{t} \\
			& =  \lim_{t \to 0} \frac{t h^2k^2}{t^2h^2k^2 + (k - h)^2}
		\end{align*}
		Now assuming that $(k - h)^2 \neq 0$ then $\lim_{t\to 0} t^2h^2k^2 + (k - h)^2 = (k - h)^2 \neq 0$ so the limit becomes:
		\begin{align*}
			& =  \lim_{t \to 0} \frac{t h^2k^2}{t^2h^2k^2 + (k - h)^2} \\
			& =  \frac{\lim_{t \to 0} t h^2k^2}{\lim_{t \to 0} t^2h^2k^2 + (k - h)^2} \\
			& =  0
		\end{align*}
		Assuming that $(k - h)^2 = 0$ then $k = h$ so the limit becomes:
		\begin{align*}
			& =  \lim_{t \to 0} \frac{t h^2k^2}{t^2h^2k^2 + (k - h)^2} \\
			& =  \lim_{t \to 0} \frac{t k^4}{t^2k^4} \\
			& =  \lim_{t \to 0} \frac{1}{t}
		\end{align*}
		Which does not exist. Therefore the directional derivatives only exist when $k \neq h$. 
		\item Since $D_1 f$ and $D_2 f$ are the directional derivatives of elementary unit vectors then $k \neq h$, and thus both of these will be zero.
		\item The function is not differentiable since some of the directional derivatives don't exist.
		\item The function is continuous.
	\end{enumerate}

	\item Munkres Question 4
	\begin{enumerate}[a)]
		\item Note the function is to be written as follows
		\[
		 f((x,y)) =
		  \begin{cases}
		   \frac{x^3}{x^2 + y^2} & \text{if } (x, y) \neq 0 \\
		   0       & \text{if } (x, y) = 0
		  \end{cases}
		\]
		Let $u = (h, k)$ be an arbitrary, non-zero vector. The directional derivative of $f$ at origin with respect to $u$ will then be:
		\begin{align*}
			f'(0; u) & =  \lim_{t \to 0} \frac{f(0 + tu) - f(0)}{t} \\
			& =  \lim_{t \to 0} \frac{t^3 h^2}{t^2(h^2 + k^2)} \\
			& =  \lim_{t \to 0} \frac{t h^2}{h^2 + k^2}
		\end{align*}
		Which is always zero
		\item All directional derivatives exist, therefore $D_1 f$ and $D_2 f$ must exist.
		\item The function is differentiable since all of the partial derivatives are continuous.
		\item The function is continuous since it's differentiable.
	\end{enumerate}

	\item Additional work, question 1
	\begin{enumerate}[a)]
		\item Note that the function $f:\mathbb{R}^2 \to \mathbb{R}$ can be rewritten like so:
		\[
		 f((x,y)) =
		  \begin{cases}
		   \frac{x|y|}{\|(x,y)\|} & \text{if } (x, y) \neq 0 \\
		   0       & \text{if } (x, y) = 0
		  \end{cases}
		\]
		Let $u = (h, k)$ be an arbitrary, non-zero vector. The directional derivative of $f$ at origin with respect to $u$ will then be:
		\begin{align*}
			f'(0; u) & =  \lim_{t \to 0} \frac{f(0 + tu) - f(0)}{t} \\
			& =  \lim_{t \to 0} \frac{t h |t k|}{\|tu\|}\frac{1}{t} \\
			& =  \lim_{t \to 0} \frac{t |t| h |k|}{t|t|\|u\|} \\
			& =  \frac{h |k|}{\|u\|}
		\end{align*}
		Which is defined for all $u \neq 0$. Therefore all the directional derivatives exist for $f$.
		\item If $f$ were differentiable at the origin then $f(0; u) = ah + bk$ for some constant $a,b \in \mathbb{R}$, however there are no such constants that satisfy $ah + bk = \frac{h |k|}{\|u\|}$. Therefore $f$ is not differentiable at the origin.
	\end{enumerate}

	\item Additional work, question 2
	\begin{enumerate}[a)]
		\item $f(1,1) = 1^3 + 1^3 + 1 -  3 = 0$
		\item Assuming that $f_y(x,y)$ means $D_2 f(x,y) = f((x,y); e_2)$, then we have;
		\begin{align*}
			f'((x, y); e_2) & =  \lim_{t \to y} \frac{f((x, t)) - f((x,y))}{t-y} \\
			& =  \lim_{t \to y} \frac{x + t^3 + t - 3 - x - y^3 - y + 3}{t-y} \\
			& =  \lim_{t \to y} \frac{t^3 - y^3 + t - y}{t-y} 
		\end{align*}
		Application of l'Hopital's rule gives:
		\begin{align*}
			\lim_{t \to y} \frac{t^3 - y^3 + t - y}{t-y} & = \lim_{t \to y} \frac{3t^2 + 1}{1} \\
			& = 3y^2 + 1
		\end{align*}
		Which is independent of $x$, so we only need to consider $y \in [0, 2]$. Since $D_2 f((x,y))$ is an up facing parabola with vertical shift of +1, it will be greater than or equal than one over the entire number line. Therefore $D_2 f ((x,y)) \geq 1$ for $(x,y) \in [0,2]^2$
		\item From the above, we have that the functions $\phi(x) = f((x, 0))$ and $\phi'2(x) = f((x, 2))$ have no critical points, therefore we only need to show that $\phi(1-h),\phi(1+h) \leq -1$ and $\phi'(1-h),\phi'(1-h) \geq 1$ for some $h$. Taking $h = 1$ yields:
		$$ \phi(0) = -3 \leq -1$$
		$$ \phi(2) = -1 \leq -1$$
		$$ \phi'(0) = 7 \geq 1$$
		$$ \phi'(2) = 9 \geq 1$$
		\begin{align*}
		\end{align*}
	\end{enumerate}


\end{enumerate}
\end{document}
