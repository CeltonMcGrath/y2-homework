\documentclass[letterpaper,10pt]{article}
\usepackage[T1]{fontenc}
\usepackage[utf8]{inputenc}
\usepackage{lmodern}
\usepackage[english]{babel}
\usepackage{amsmath}
\usepackage{amsfonts}
\usepackage{fullpage}
\usepackage{enumerate}
\usepackage{setspace}
\usepackage{ amssymb }

% \def\pball{\mathrel{%
%     \mathchoice{\PBALL}{\PBALL}{\scriptsize\PBALL}{\tiny\PBALL}%
% }}
% \def\PBALL{{%
%     \setbox0\hbox{B}%
%     \rlap{\hbox to \wd0{\hss|\hss}}\box0
% }}

\allowdisplaybreaks
\newcommand{\tinyquot}[1]{\begin{center}{\footnotesize #1}\end{center}}
\newcommand*{\QEDA}{\hfill\ensuremath{\blacksquare}}
\addtolength{\belowdisplayskip}{-25mm}
\begin{document}
\title{Assignment 2 - MAT257}
\author{David Knott \\  Student \#999817685}
\date{September 18, 2013}
\maketitle
\begin{enumerate}
	\item Munkres, \S3, Question 8

	\item Munkres, \S3, Question 9
	\item Additional Work, Question 1

	Consider first arbitrary unions of open sets. Suppose $S$ is the set of sets we're unioning. Since for every $x \in \cup S$ there is a $s \in S$ such that $x \in s$, and since all $s \in S$ is open, there exists an $\epsilon > 0$ such that $B(x, \epsilon) \subset s$. Since $s \subseteq \cup S$ this implies for every $x \in \cup S$ there's an $\epsilon > 0$ such that $B(x, \epsilon) \subset \cup S$. Therefore, arbitrary unions of open sets are open.

	For finite intersections we note that if $x \in \cap S$ then for all $s \in S$ we have $x \in s$. Since all these sets are open, for all $s \in S$ there exists an $\epsilon > 0$ such that $B(x, \epsilon) \subset s$. If we take the minimum of all these epsilons, call it $\epsilon'$, we have for all $s \in S$ that $B(x, \epsilon') \subset s$ since $\epsilon'$ is smaller than or equal to any of the epsilons. This is proof that there exists an $\epsilon > 0$ such that for all $x \in \cap S$ then $B(x, \epsilon) \subset \cap S$, since we simply take $\epsilon '$.

	If we allow for arbitrary intersections of open sets it's possible to create a closed set. Consider for example the set of all sets that contain $0$, which we will call $S$. Since the inclusion of $0$ in a set does not imply the inclusion of any other points, for any point $x \neq 0$ there's a set in $S$ that does not contain $x$. This implies that $x \notin \cap S$. So the only vector in the intersection is $0$, which makes it a closed set.

	To prove the similar assertions for intersections and unions of closed sets, note that $\overline{S \cup S'} = \overline{S} \cap \overline{S'}$ and $\overline{S \cap S'} = \overline{S} \cup \overline{S'}$. Applying the rules we just proved we can see that arbitrary unions of the complement of sets is the same as arbitrary intersections of the set, so it directly implies that arbitrary intersections of closed sets are closed. The same goes for finite unions.

	\item Additional Work, Question 2

	Let $X$ and $Y$ be metric spaces with metrics $d_X$ and $d_Y$ respectively. Let $f : A \rightarrow Y$ where $A \subset X$. 

	If $x_0$ is an isolated point, then for some $\delta > 0$ there are no points in $B(x_0, \delta)$ other than $x_0$. Therefore, $f$ is continuous at $x_0$ because the assertion $\forall x \in X\ .\ d_X(x, x_0) < \delta \implies d_Y(f(x), f(x_0))$ is vacuously true.

	Suppose for the rest of this section that $x_0$ is not an isolated point. In that case, $x_0$ must be a limit point.

	Assume that $f$ is continuous at $x_0$. This means, by definition, that for any open subset $V$ of $Y$ containing $f(x_0)$ then there is an open subset $U$ of $A$ containing $x$ such that $f(U) \subset V$. The statement ``$f(x) \to f(x_0)$ as $x \to x_0$'' means for any open subset $V$ of $Y$ containing $f(x_0)$ then there is an open subset $U$ of $X$ containing $x$ such that $x \neq x_0 \wedge x \in U \cap A \implies f(x) \in V$.
	% $$ d_X(x, x_0) < \delta \implies d_Y(f(x), f(x_0)) < \epsilon $$
	% Now note the assertion ``$f(x) \to f(x_0)$ as $x \to x_0$'' means for any $\epsilon > 0$, there exists a $\delta > 0$ such that for all $x \in X$:
	% $$0 < d_X(x, x_0) < \delta \wedge x \in A \implies d_Y(f(x), f(x_0)) < \epsilon $$
	% since $0 < d_X(x, x_0) < \delta \wedge x \in A$ implies $d_X(x, x_0) < \delta$ we simply substitute it into the antecedent of the continuity assumption to prove the limit.

	% For the converse, we assume $f(x) \to f(x_0)$ as $x \to x_0$ and try to show continuity. Remember that we assume for all $\epsilon > 0$, there exists a $\delta > 0$ such that for all $x \in X$:
	% $$0 < d_X(x, x_0) < \delta \wedge x \in A \implies d_Y(f(x), f(x_0)) < \epsilon $$
	% And try to prove for all $\epsilon > 0$, there exists a $\delta > 0$ such that for all $x \in X$:
	% $$ d_X(x, x_0) < \delta \implies d_Y(f(x), f(x_0)) < \epsilon $$
	% In order to prove this we must show that if $x = x_0$ or $x \notin A$ then $d_Y(f(x), f(x_0)) < \epsilon$, since the limit assumption covers the cases when both of those are true. If $x \notin A$ we can't prove that $d_Y(f(x), f(x_0)) < \epsilon$, but we can avoid this case by choosing a smaller delta that does not include any points not in $A$. Since $x_0$ is a limit point, we can take the $\delta$ that includes 

	% To prove the above we need only consider $x \in A$, since if $x \notin A$ the implication is true since the antecedent is false. Likewise, we can throw out the case that $x = x_0$ since $\delta_X(x_0, x_0) = 0$. So, assuming $x \in A$ and $x \neq x_0$, from the assumption of continuity for any $\epsilon > 0$, there exists a $\delta > 0$ such that:
	% $$ d_X(x, x_0) < \delta \implies d_Y(f(x), f(x_0)) < \epsilon $$
	% Throwing in our restrictions on $x$ yields:
	% $$0 < d_X(x, x_0) < \delta \wedge x \in A \implies d_Y(f(x), f(x_0)) < \epsilon $$
	% So continuity implies $f(x) \to f(x_0)$ as $x \to x_0$

	\item Additional Work, Question 3

\end{enumerate}
\end{document}